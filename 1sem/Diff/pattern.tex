\documentclass[12pt]{article}
\usepackage[russian]{babel}
\usepackage[utf8]{inputenc}
\textheight=235mm \textwidth=170mm \hoffset=-20mm \voffset=-20mm

\begin{document}
	\topskip=5cm
	
	\begin{center}
		\fontsize{30}{12pt}{\textbf{\emph{\textit{\underline{Доклад}}}}}\\ \bigskip \bigskip
		\fontsize{30}{12pt}{\textbf{\emph{\textit{\underline{на тему}}}}}\\ \bigskip \bigskip
		\textit{\fontsize{20}{12pt}{\textbf{\emph{\textit{\underline{\lq\lq Дифференцирование элементарных функций\rq\rq}}}}}}
	\end{center}
	
	\vspace{5cm}
	\begin{flushright}
		{\fontsize{16}{12pt}\emph{\textbf{Дробышев Андрей, 372}}}
	\end{flushright}
	
	
	\topskip=0mm
	\newpage

	Привет, меня зовут Андрей, и я люблю дифференцировать. И сегодняшняя наша цель -- выражение
	
	\begin{center}
		$#$.
	\end{center}
	\bigskip
	
	Ну что ж, приступим. Пользуясь нехитрыми правилами дифференцирования, из исходного выражения получим следующее:
	 
	\begin{center}
		$@$.
	\end{center}
	\bigskip
	
	Не слишком удобочитаемо, не находите? Однако могущество матана поистине безгранично, и после консультации Дмитрия Владимировича мы-таки сумели упростить этот страшный результат:
	
	\begin{center}
		$%$.
	\end{center}
	\bigskip
	
	Так-то лучше. Дальнейшие очевидные преобразования оставляем любознательному читателю в качестве упражнения.
	
	\bigskip
	
	Спасибо за внимание.

\end{document}